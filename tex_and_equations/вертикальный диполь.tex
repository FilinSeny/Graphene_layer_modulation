\documentclass[a4paper, 12pt]{article}
\usepackage[russian]{babel}
\usepackage[T2A]{fontenc}
\usepackage[utf8]{inputenc}


\usepackage{xcolor}

\usepackage{hyperref}
\definecolor{linkcolor}{HTML}{44D663} % цвет ссылок
\definecolor{urlcolor}{HTML}{031A9B} % цвет гиперссылок

\hypersetup{pdfstartview=FitH,  linkcolor=black,urlcolor=urlcolor, colorlinks=true}

\usepackage{graphicx}

\usepackage{multirow}
\usepackage{multicol}

\usepackage{amsmath} 
\usepackage{listings}
\usepackage{xcolor}
\usepackage{color}
\usepackage{amsmath}

\definecolor{dkgreen}{rgb}{0,0.6,0}
\definecolor{gray}{rgb}{0.5,0.5,0.5}
\definecolor{mauve}{rgb}{0.58,0,0.82}
\lstset { %
	language=C++,
	backgroundcolor=\color{black!5}, % set backgroundcolor
	basicstyle=\footnotesize,% basic font setting
	tabsize=3,
	breaklines=true,
	breakatwhitespace=true,
	numbers=left,
	% Стиль литералов
}

\usepackage{pgfplots}
\pgfplotsset{compat=1.9}



\usepackage{amsmath}

\usepackage{amsmath, amsfonts}



\begin{document}
	\newpage 
	Для вертикального диполя в трехслойной среде получаем ур-я
	
	для X ур-я вида:
	
	\begin{equation}
		\begin{cases}
			X_1 = d_1 e^{-n_1 z}, z > 0 \\
			X_2 = d_2 e^{-n_2 z} + c_2 e^{n_2 z}, 0 < z < z_1 = H \\
			X_3 = c_3 e^{-n_3 z}, z < H
			
		\end{cases}		
	\end{equation}
	
	
	При $z = 0$ имеем условия вида:
	
	\begin{equation}
		\begin{cases}
			d_1 = d_2 + c_2 \\
			-n_1 d_1 = -n_2 d_2 + n_2 c_2 + i \omega \delta \frac{1}{k_2^2}n_2 (\gamma_2 - \beta_2)
			
		\end{cases}
	\end{equation}
	
	При $z = z_1 = H$:
	
	\begin{equation}
		\begin{cases}
			d_2 e^{-n_2 z_1} + c_2 e^{n_2 z_1} = c_3 e^{n_3 z_1} \\
			-n_2 d_2 e^{-n1 z_1} + n_2 c_2 e^{n_2 z_1} = n_3 c_3 e^{n_3 z_1}
		\end{cases}
	\end{equation}
	
	Решая получим:
	
	\begin{equation}
		\begin{aligned}
			c_3  = \frac{2 c_2}{1 + \frac{n_3}{n_2}} e^{z_1(n_2 - n_3)}\\
			d_2 = c_2 e^{2 z_1 n_2} \frac{n_2 - n_3}{n_2 + n_3} \\
			d_1 = c_2 (1 + e^{2 z_1 n_2} (\frac{n_2 - n_3}{n_2 + n_3})) \\
			c_2 = i \omega \delta \frac{n_2}{k_2^2} (\gamma_2 - \beta_2)
		\end{aligned}
	\end{equation}
	
	\newpage
	
	для V получаем ур-я:
	\begin{equation}
		\begin{cases}
			V_1^* = \beta_1^* e^{-n_1 z}, z > z_0 \\
			V_1 = \beta_1 e^{-n_1 z} + \gamma_1 e^{n_1 z}, 0 < z < z_0 \\
			V_2 = \beta_2 e^{-n_2 z} + \gamma_2 e^{n_2 z}, z_1 < z < 0 \\
			V_3 = \gamma_3 e^{n_3 z}, z < z_1
		\end{cases}
	\end{equation}
	
	и набор краевых условий
	\newline
	
	
	при $z = z_0$:
	
	\begin{equation}
		\begin{cases}
			\beta_1^8 e^{-n_1 z_0} = \beta_1 e^{-n_1 z_0} + \gamma_1 e^{n_1 z_0} \\
			
			-n_1 \beta_1^* e^{-n_1 z_0} = -n_1 \beta_1 e^{-n_1 z_0} + n_1 \gamma_1 e^{n_1 z_0} + 2
		\end{cases}
	\end{equation}
	
	
	при $z = 0$:
	
	\begin{equation}
		\begin{cases}
			\frac{1}{k_1^2} (-n_1 \beta_1 + n_1 \gamma_1) = \frac{1}{k_2^2} (-n_2 \beta_2 + n_2 \gamma_2) \\
			
			\beta_1 + \gamma_1 - \beta_2 - \gamma_2 = i  \omega \delta (d_2 + c_2) \\
			
		\end{cases}	
	\end{equation}
	
	
	при $z = z_1 = H$:
	
	\begin{equation}
		\begin{cases}
			\frac{1}{k_2^2}(\beta_2 (-n_2) e^{-n_2 z_1} + \gamma_2 n_2 e^{n_2 z_1}) = \frac{1}{k_3^2}\gamma_3 n_3 e^{n_3 z_1} \\
			
		\beta_2 e^{-n_2 z_1} + \gamma_2 e^{n_2 z_1} = \gamma_3 e^{n_3 z_1} \\
		\end{cases}
	\end{equation}
	
	
	решая получим:
	
	\begin{equation}
		\begin{aligned}
			\beta_1 = \beta_1^* - \gamma_1 e^{2 n_1 z_0} \\
			\gamma_1 = \frac{2}{n_1} e^{-n_1 z_0} \\
			\beta_2 = \frac{1}{2} (q_{21} \beta_1 + \gamma_1 p_{21} - L) \\
			\gamma_2 = \frac{1}{2} (\beta_1 q_{21} + \gamma_1 p_{21} + L) \\
			L = i \omega \delta (d_2 + c_2) \\
			p_{21} = 1 - \frac{k_2^2 n_1}{k_1^2 n_2} \\
			q_{21} = 1 + \frac{k_2^2 n_1}{k_1^2 n_2} \\
			\gamma_3 = \beta_2 e^{z_1 (-n_3 - n_2)} + \gamma_2 e^{n_2 - n_3} \\
			\beta_2 = \gamma_2 \frac{n_2 k_3^2 + n_3 k_2^2}{n_2 k_3^2 - n_3 k_2^2} e^{2 z_1 n_2}  \\
		\end{aligned}
	\end{equation}
	
	Выразим L из 4:
	
	\begin{equation}
		L = - \omega^2 \delta^2 \frac{1}{k_1^2} n_2 (\gamma_2 - \beta_2) (1 + e^{2 z_1 n_2 } \frac{n_2 - n_3}{n_1 + n_2})
	\end{equation}
	
	пусть:
	
	$$R = - \omega^2 \delta^2 \frac{1}{k_1^2} n_2 (1 + e^{2 z_1 n_2 } \frac{n_2 - n_3}{n_1 + n_2})$$
	
	и 
	
	$$N =  \frac{n_2 k_3^2 + n_3 k_2^2}{n_2 k_3^2 - n_3 k_2^2} $$
	
	Тогда L примет вид:
	
	$$L = Q ((\beta_1^* - \gamma_1 e^{2n_1 z_0} ) p_{21} + \gamma_1 q_{21}) $$
	где
	$$Q = \frac{R (1 - Ne^{2 z_1 n_2})}{2 (1 - \frac{R}{2}(1 - N e^{2 z_1}))}$$
	
	тогда 
	
	\begin{equation}
		\begin{aligned}
			\beta_1^* (q_{21} - p_{21} N e^{2 z_1 n_2}) = \\
			\gamma_1 (q_{21} 	(Ne^{2z_1 n_2} + e^{2 n_1 z_0}) - p_{21} (N e^{2 z_1 n_2 - 2 n_1 z_0} + 1) +  
			Q(Ne^{2 z_1 n_2} - 1) (q_{21} - p_{21} e^{2 n_1 z_0}) )\\
		\end{aligned}
	\end{equation}
	

	
	
	 
\end{document}